\documentclass[book,paper=b5j,head_space=2cm,foot_space=2cm,fontsize=9pt]{jlreq}

\usepackage{graphicx}
\usepackage{caption} % 画像説明文用
\renewcommand{\figurename}{Figure}

% ソースコード表示関連
\usepackage{listings}
\lstdefinestyle{codestyle}{
  frame=single,
  backgroundcolor=\color{white},
  basicstyle=\ttfamily\small,
  breaklines=true,
  captionpos=b,
  keepspaces=true,
  showspaces=false,
  showstringspaces=false,
  showtabs=false,
}
\lstset{style=codestyle}

\setlength{\fboxsep}{0pt}

\usepackage[pdfencoding=auto,hidelinks]{hyperref}

\usepackage{xcolor}
\definecolor{highlight}{rgb}{0.85,0.85,0.85}

\usepackage[overlay]{textpos}
\setlength{\TPHorizModule}{1mm}
\setlength{\TPVertModule}{1mm}


\usepackage{mdframed}
\newmdenv[
  backgroundcolor=gray!10,
  linecolor=gray!60,
  linewidth=1pt,
  roundcorner=5pt,
  skipabove=\baselineskip,
  skipbelow=\baselineskip,
]{columnbox}

% 画像の表示
\newcommand{\includeimage}[3]{
  \begin{figure}
    \centering
    \includegraphics[width=#1\textwidth]{#2}
    \captionof{figure}{#3}
  \end{figure}
}

% 本タイトル定義
\newcommand{\bookTitle}{本タイトル}

% 著者名定義
\newcommand{\authorExample}{著者名}

% 章タイトル定義
\newcommand{\chTitleExample}{章タイトル}

% 奥付情報
\newcommand{\bookPrinting}{印刷会社}
\newcommand{\bookAuthor}{\authorExample}
\newcommand{\bookPublisher}{発行元}
\newcommand{\bookEditor}{編集者名}
\newcommand{\bookIllustrator}{イラストレーター名}
\newcommand{\bookDesigner}{デザイナー名}
\newcommand{\bookDate}{Dec 15, 2010}
\newcommand{\bookFirstDate}{2010年12月15日}
\newcommand{\bookUrl}{https://example.com}
